\documentclass{article}
\usepackage{multirow} 
\begin{document}
	Objetivo:
	EL objetivo de visualizar los riesgos que conlleva el desarrollo de un proyecto radica en la importancia de ser realista con un proyecto, con sus alcances, sus limites, su potencial y especialmente con sus tiempos. \\
	Para lograr este objetivo de visualizar de mejor manera los riesgos potenciales que se daran durante el desarrollo del presente proyecto se realizo una tabla.\\ 
	La tabla que se muestra a continuacion describe los riesgos de la elaboracion del presente proyecto, los riesgos de un proyecto son una parte crucial a tener en cuenta, ya que si no se conocen los problemas que pueden ser ocasionados por los miembros del equipo, el proyecto puede simplemente no despegar.	\\
	\\
	\\
	\\
\begin{tabular}{|p{2cm}|p{2cm}|p{2cm}| p{2cm}| p{2cm}| p{2cm}|}
\hline 	
Risk ID& Risk Statement& Impact on project&Probability on ocurrane & Risk exposure  \\\hline 

1& Poco conocimiento de las tecnologias & Retrasos& 70 por ciento & 40 por ciento \\\hline 

2& Poco tiempo para entregar& Defectos en el proyecto& 20 por ciento & 10 por ciento \\\hline 

3& Imposibilidad de reunirnos& Deficiente comunicacion& 80 por ciento & 50 por ciento \\\hline 

4& Enfermedades& NO trabajar por alguna enfermedad (como el covid)& 50 por ciento & 0 por ciento \\\hline 

5& Desercion escolar& Que algun miembro del equipo deserte de la carrera& 90 por ciento & 0 por ciento \\\hline 

6& Desidia& Que algun miembro del equipo tenga flojera y no quiera hacer su parte& 40 por ciento & 5 por ciento \\\hline 

7& Falta de herramientas & Herramientas no adecuadas para trabajar& 10 por ciento & 20 por ciento \\
\hline 









	
\end{tabular}
\\
\\
\\

Los riesgos ya estan definidos, pero ¿como podemos permearlos? \\

Bueno, algunos de los riesgos son inamovibles, sin embargo los que a nuestro criterio se pueden solucionar son los riesgos numero 1 y numero 6, ya que son elementos que estan en nuestras manos. \\
Para el riesgo 1 la solucion es simple, no dejar de aprender, nunca vamos a saber completamente una tecnologia, pero podemos aprender cosas nuevas a diario.\\
Para el riesgo 2 la solucion parece bastante obvia, sin embargo muchas veces es complicado encontrar la motivacion para hacer las cosas, pero una solucion podria ser siempre intentar hacer algo que nos apasione, no siempre se puede, pero es una medida paliativa\\

En conclusion un proyecto siempre tendrà riesgos, no importa la manera o el esquema en el que se trabaje, ya que por mucha tecnologia que conlleve un proyecto, los desarrolladores son humanos, con imperfecciones.

\end{document}

