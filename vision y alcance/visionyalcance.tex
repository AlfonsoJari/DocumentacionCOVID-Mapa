
\documentclass{article}
\usepackage{graphicx}
% Esto es para poder escribir acentos directamente:
\usepackage[latin1]{inputenc}
% Esto es para que el LaTeX sepa que el texto est� en espa�ol:
\usepackage[spanish]{babel}
% Paquetes de la AMS:
\usepackage{amsmath, amsthm, amsfonts}
\usepackage[left=3cm,right=3cm,top=3cm,bottom=3cm]{geometry} 


\usepackage{graphicx}

\def\RR{\mathbb{R}}
\def\ZZ{\mathbb{Z}}

\newcommand{\abs}[1]{\left\vert#1\right\vert}

\DeclareMathOperator{\Jac}{Jac}
\title{VISISON Y ALCANCE DEL PROYECTO}
\author{EQUIPO 1\\

	\\
  \LARGE DEP. Ingenieria de Software\\
  
  \\
  
  \LARGE Maya Hernandez Alfonso Jari\\
  
  \\  
  
  \LARGE Campos Mora Luis Edel\\
    
  \\
  
  \LARGE Alonso Figueroa Alejandro\\  
  
  
  \\
  \LARGE Teran Juarez Cristian\\
  
    
  \\
  \LARGE Ramirez Ruiz Alexi\\
    
  \\
  
  \LARGE Gomez Montes Cristian Leopoldo\\
  
  \\
   
}
\begin{document}
\maketitle
\newpage

\title 
\Huge{INDICE GENERAL} 
\large \\


 1 INTRODUCCION...............3\\


 1.1 VISION...................3\\


 1.1.1 DECLARACION DE LA VISION.................3\\


 1.1.2 REQUERIMIENTOS FUNCIONALES................3\\


 1.1.3 SUPOSICIONES Y DEPENDENCIAS.......................4\\


 2 ALCANCE DE LA VISION...............4\\


 2.1 ALCANCE DE LA VERSION INICIAL..............5\\


 2.1.1 ALCANCE DE LAS VERSIONES POSTERIRORES..........5\\
 
\includegraphics[scale=1.5]{uv_logo}
\centering %Figura centrada.
\newpage

\section{INTRODUCCION}
\vspace{0.7 cm}
\Large Proyecto enfocado en el analisis de informacion acerca del COVID-19. Mostraremos informacion basada sobre la base de datos donde se mostrara la informacion necesaria sobre la actual situacion necesaria para el cliente\\
 
 
\subsection{VISION}\label{sec:nada}
\begin{verse}
\vspace{0.5 cm}
Esta seccion establece una vision a largo plazo del sistema que se
construira. Esta vision proporcionara
el contexto para la toma de decisiones a lo largo del ciclo de vida
del desarrollo del producto.\\

\end{verse}

\subsubsection{DECLARACION DE LA VISION}\label{sec:nada2}
  
\begin{verse}
\Large COVID-19 ESTADISTICAS. es un proyecto enfocado en el analisis y exposicion de datos relacionados a la pandemia del COVID-19. Refinamos los terminos de busqueda para aumentar el exito y la facilidad al navegar por el sitio web. Todo debera estar conectado a la base de datos de la plataforma para asi mismo poder dar un mejor manejo en la web a crear donde la informacion estara clara y concisa.
Las mejoras pueden identificarse a medida que avanza el proyecto.\\
\end{verse}

\subsubsection{REQUERIMIENTOS FUNCIONALES Y NO FUNCIONALES}
\begin{verse}
\Large Requerimientos funcionales:	\\


-Inicio de sesion con correo electronico\\


-mostrar un mapa con informacion sobre el covid19\\


-Que la base de datos guarde nombre de usuario y su contrase�a para el inicio de sesion\\


-Mostrar las estadisticas disponibles\\
	

-Mostrar cantidad de casos de contagio en el mapa\\


-Generar sin problemas un mapa con la informacion de la base de datos del covid19\\


Requerimientos no funcionales:\\


-Adaptabilidad (Debe admitir la capacidad de trabajar con diferentes almacenes de datos)\\

\end{verse}

\subsubsection{ SUPOSICIONES Y DEPENDENCIAS}
\begin{verse}
\Large   Las siguientes tecnolog�as se han identificado como limitaciones tecnicas el desarrollo de la pagina web debe ser desarrollada con las siguientes tecnologias:\\
	-Angular\\
	-Nodejs\\
	-Postgres\\
	-Apis\\
	-Spring\\
	-Github\\
	-Html\\
	
\end{verse}

\section{ALCANCE Y LIMITACIONES}
\begin{verse}
\Large El alcance del proyecto define el concepto y el alcance de la soluci�n propuesta.\\


\end{verse}
\subsection{ALCANCE DE LA VERSION INICAL}\label{sec:nada}
\begin{verse}
\Large Se ha planificado un lanzamiento inicial que sirva como prueba de concepto para el producto final.
producto. Esta versi�n inicial (R1) consistir� en una demostraci�n funcional del marco en
que se construir� el resto del proyecto.\\

--Dise�o de la base de datos completa \\


--Mapa de guia funcional y didactico\\


--Instrucciones de inicio de manejo del mapa\\


--Ayuda si es necesaria al iniciar sesion \\


--Rama de preguntas frecuentes para la ayuda del cliente al necesitar ayuda con problemas que sean comunes al usar el mapa la primera vez\\


--Integracion de una base de datos\\


--Pruebas para la pagina web en funcionamiento \\

\end{verse}
\subsubsection{ALCANCE DE LAS VERSIONES POSTERIORES}\label{sec:nada2}
\begin{verse}
\Large- a�adir informacion actualizada\\

	- Pruebas programadas para el testing de la pagina web
	
	- Reaccion rapida de la pagina al seleccionar una pesta�a
	
	- Datos verificados de la base de datos
	
	- Testing las diferentes pesta�as de la pagina web 

\end{verse}


\end{document}
